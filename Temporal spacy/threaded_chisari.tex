\begin{filecontents*}{threaded_chisari.tex}
    \documentclass{article}
    \usepackage{amsmath,amssymb,amsthm}
    
    %-----------------------------------------------------------------
    % theorem‑like environments
    %-----------------------------------------------------------------
    \newtheorem{definition}{Definition}
    \newtheorem{proposition}{Proposition}
    \newtheorem{lemma}{Lemma}
    \newtheorem{corollary}{Corollary}
    
    \title{Threaded Chisari Quivers:\\
           a Planar Cover Digraph Presentation of Quasi‑Orders}
    \author{}
    \date{}
    
    \begin{document}
    \maketitle
    
    %=================================================================
    \section{The construction}
    
    \begin{definition}[Data]\label{def:data}
    Fix
    \begin{itemize}
      \item a totally ordered set $(J,<)$ (think ``time‑levels'');
      \item a set $T$ (think ``threads'').
    \end{itemize}
    Define the \emph{vertex set}
    \[
      Q_0 \;:=\; T \times J, 
      \qquad (t,j)\in Q_0.
    \]
    \end{definition}
    
    \begin{definition}[Threaded Chisari quiver]\label{def:quiver}
    The \emph{threaded Chisari quiver} is the digraph
    \[
      Q \;=\; (Q_0,Q_1,\;s,t)
    \]
    whose arrows are the union of
    \begin{itemize}
    \item \textbf{thread covers}  
          \[
            (t,j)\xrightarrow{\;\;\;} (t,j') 
            \quad\text{iff } j<j' \text{ and } j' \text{ is the immediate successor of } j;
          \]
    \item \textbf{equivalence two‑cycles}  
          \[
            (t,j)\xrightleftharpoons{\;\;\;} (t',j)
            \quad\text{for all } t\neq t' \text{ and fixed } j.
          \]
    \end{itemize}
    \end{definition}
    
    %=================================================================
    \section{From quiver to quasi‑order}
    
    \begin{definition}[Path preorder]\label{def:preorder}
    Declare $(t,j)\preccurlyeq(t',j')$ iff
    there exists a (possibly empty) directed path in $Q$  
    from $(t,j)$ to $(t',j')$.  
    Write $\sim$ for the associated equivalence:
    \(
      x\sim y \;\Leftrightarrow\; x\preccurlyeq y \text{ and } y\preccurlyeq x.
    \)
    \end{definition}
    
    \begin{proposition}\label{prop:preorder}
    $(Q_0,\preccurlyeq)$ is a quasi‑order and 
    \(
      (t,j)\sim(t',j') \Longleftrightarrow j=j'.
    \)
    \end{proposition}
    
    \begin{proof}
    Reflexivity is witnessed by empty paths.
    Transitivity is closed under concatenation of paths; hence $\preccurlyeq$ is a quasi‑order.
    
    If $j=j'$ the two‑cycle yields paths both ways, so $\sim$ holds.
    Conversely assume $(t,j)\sim(t',j')$.  
    Any path that leaves level $j$ must change $j$, but the only arrows that
    change the $j$–coordinate are thread covers, which are \emph{strictly} one‑way.
    Thus a round‑trip path $(t,j)\rightsquigarrow(t',j')\rightsquigarrow(t,j)$ 
    cannot change the $j$–coordinate; hence $j=j'$.
    \end{proof}
    
    %=================================================================
    \section{Minimality (Chisari property)}
    
    \begin{lemma}\label{lem:irreducible}
    Every thread cover arrow is irreducible 
    (no $z$ with $(t,j)\prec z\prec (t,j+1)$),
    and for $j$ fixed the two‑cycle arrows generate the entire $\sim$–class.
    \end{lemma}
    
    \begin{proof}
    Inside a fixed thread $t$ there is no vertex of
    intermediate $j$ between $j$ and its successor, so the cover cannot factor.
    Two‑cycle arrows connect all vertices of a row;
    any composite of such arrows remains within the row, 
    so the row is exactly one equivalence class.
    \end{proof}
    
    \begin{proposition}[Threaded quiver is already Chisari]\label{prop:chisari}
    $Q_1$ consists precisely of the irreducible arrows of $(Q_0,\preccurlyeq)$ 
    together with the invertible arrows inside each $\sim$–class.
    \end{proposition}
    
    \begin{proof}
    By Lemma~\ref{lem:irreducible} each listed arrow is irreducible or invertible, 
    and every other arrow factors through them, so $Q$ is the cover digraph.
    \end{proof}
    
    %=================================================================
    \section{Quotient poset and Hasse diagram}
    
    \begin{definition}[Quotient]\label{def:quotient}
    Let $\pi:Q_0\to J$ be the projection $(t,j)\mapsto j$.
    Equip $J$ with the order inherited from $<$.
    \end{definition}
    
    \begin{proposition}\label{prop:quotient}
    $\pi$ is the preorder quotient map:
    \(
      (t,j)\sim(t',j') \Longleftrightarrow \pi(t,j)=\pi(t',j').
    \)
    Moreover the induced order on $J$ is linear and its Hasse diagram is
    \[
      0<1<2<\cdots \quad\text{(or any finite/ countable initial segment)}.
    \]
    \end{proposition}
    
    \begin{proof}
    The first claim is Proposition~\ref{prop:preorder}.
    If $j<j'$ are adjacent, choose any thread $t\in T$;
    the thread cover $(t,j)\to(t,j')$ witnesses $\pi^{-1}(j)\prec\pi^{-1}(j')$,
    so $j<j'$ in the quotient.  No other covers exist by construction,
    hence these are the Hasse edges of the quotient poset.
    \end{proof}
    
    \begin{corollary}
    Collapsing each $\sim$–class in $Q$ yields exactly the Hasse diagram of the chain $(J,<)$.
    \end{corollary}
    
    %=================================================================
    \section{Planar embedding}
    
    \begin{proposition}[Straight‑line planarity]\label{prop:planar}
    There exists an embedding of $Q$ in the plane such that
    \begin{itemize}
      \item all thread covers are drawn as vertical segments,
      \item all equivalence two‑cycle arrows are drawn as horizontal segments,
      \item no two edges cross.
    \end{itemize}
    \end{proposition}
    
    \begin{proof}
    Place vertex $(t,j)$ at the integer lattice point $(x,y)=(t,j)$
    (after fixing any bijection $T\to\mathbb Z$ if $T$ is infinite;
    for finite $T$ embed them in $\mathbb Z$ consecutively).
    A thread cover changes only the $y$‑coordinate by $+1$, thus is vertical;  
    horizontal arrows lie at constant $y$.
    Vertical edges of distinct threads live in disjoint $x$ positions; 
    horizontal edges lie between those vertical lines and do not intersect them.
    Hence no two edges cross.
    \end{proof}
    
    %=================================================================
    \section*{Summary}
    A threaded Chisari quiver is the cover digraph of a quasi‑order whose
    quotient by the row‑wise equivalence is a simple linear chain.
    It is already transitive‑reduced, and collapsing the equivalence classes
    automatically produces the Hasse diagram of the associated poset.
    \end{document}
    \end{filecontents*}
    