\documentclass{article}
\usepackage{amsmath,amsthm,amssymb,hyperref}
\title{Hasse\,/\,Chisari Quiver Constructions in Category--Theoretic Formality}
\author{}
\date{}

\newtheorem{definition}{Definition}
\newtheorem{proposition}{Proposition}
\newtheorem{remark}{Remark}

\begin{document}
\maketitle

\section{Preliminaries}
Let $C$ be a small \emph{thin} category (i.e.~a quasi--order), where
\begin{itemize}
  \item objects $\mathrm{Ob}(C)$ correspond to the elements of the quasi--order;  
  \item hom-sets satisfy $|\mathrm{Hom}_C(x,y)|\le 1$ and $\mathrm{Hom}_C(x,y)\neq\varnothing$ precisely when $x\le y$.
\end{itemize}
For background on quasi--orders and their relation to partial orders see~\cite{Rosenstein1982}.

\section{Skeleton, Quotient and Transitive Reduction}

\begin{definition}[Isomorphism relation]\label{def:iso}
In $C$ we write $x\cong y$ when there are morphisms
\[
  f: x\to y, \qquad g:y\to x, \qquad g\!\circ\!f = \mathrm{id}_x,\; f\!\circ\!g = \mathrm{id}_y.
\]
Because $C$ is thin this is equivalent to $x\le y$ and $y\le x$.  The equivalence classes $[x]_{\cong}$ are the \emph{strongly-connected components} of the digraph underlying $C$.
\end{definition}

\begin{definition}[Skeleton]
A \emph{skeleton} $\mathrm{Sk}(C)$ is any full subcategory containing exactly one representative of each class $[x]_{\cong}$.  It is therefore a \emph{skeletal} thin category, i.e.~a poset.
\end{definition}

\begin{definition}[Irreducible morphisms = covers]
Let $\mathsf{P}$ be a skeletal thin category.  A non-identity morphism $f\!:\!A\to B$ is \emph{irreducible} if
\[
  A < B \quad\text{and}\quad \nexists\,C\text{ with }A<C<B.
\]
Such pairs $(A,B)$ are the \emph{covering relations} of the poset.
\end{definition}

\begin{proposition}[Hasse quiver of the skeleton]
The Hasse quiver of $C$ is
\[
  Q_H = \bigl(\mathrm{Ob}(\mathrm{Sk}(C)),\,\mathrm{Irr}(\mathrm{Sk}(C))\bigr).
\]
Its underlying digraph is acyclic and its transitive closure reproduces $\mathrm{Sk}(C)$, hence --- after reinserting the equivalence classes --- the whole of $C$.
\end{proposition}

\section{The \textbf{Chisari} (mixed) Quiver}

\begin{definition}[Iso/\!Irr inside $C$]
\begin{align*}
\mathrm{Iso}(C) &= \{\,f:x\to y \mid f\text{ invertible}\},\\[2pt]
\mathrm{Irr}(C) &= \{\,f:x\to y\neq\mathrm{id}\mid f\text{ is irreducible in }C\}.
\end{align*}
\end{definition}

\begin{definition}[Chisari quiver\footnote{The term is new; the construction itself appears in the literature under names such as \emph{cover digraph of a quasi-order} or the \emph{transitive reduction with two-way edges kept}.  See \cite{Harry1995,Altomare2020}.}]\label{def:chisari}
The \emph{Chisari quiver} of a thin category $C$ is
\[
  Q_{\chi} := (\mathrm{Ob}(C),\,\mathrm{Iso}(C)\cup\mathrm{Irr}(C)).
\]
It has two-cycles inside each equivalence class and exactly the cover arrows between distinct classes.
\end{definition}

\begin{proposition}[Properties]
\leavevmode
\begin{enumerate}
  \item The sub-digraph on $\mathrm{Irr}(C)$ is acyclic.
  \item Any cycle of $Q_{\chi}$ lies entirely inside an equivalence class $[x]_{\cong}$.
  \item The mixed transitive closure of $Q_{\chi}$ recovers the whole quasi-order $C$.
\end{enumerate}
\end{proposition}

\begin{remark}[One-sentence characterisation]
A \textbf{Chisari quiver} is \emph{the canonical minimal-edge digraph that presents a quasi-order, obtained by retaining a two-cycle for each equivalent pair and keeping only the covering arrows between equivalence classes}.  In other words, it is the cover digraph of the quotient poset, expanded into cliques along the equivalence classes.
\end{remark}

\subsection*{Relation to well-quasi-orders}
The quotient functor $C\mapsto\mathrm{Sk}(C)$ is a categorical equivalence between quasi-orders and partial orders.  Consequently~\cite{Rosenstein1982}:
\begin{itemize}
  \item $C$ is a \emph{well-quasi-order (WQO)} iff its skeleton $\mathrm{Sk}(C)$ is a \emph{well-partial-order (WPO)}.
  \item Hence $C$ is a WQO \emph{iff} its Chisari quiver inherits the WQO property (the two-cycles neither create antichains nor descending sequences).
\end{itemize}
Classical examples --- Higman’s theorem on words~\cite{Higman1952}, Kruskal’s theorem on finite trees~\cite{Kruskal1960}, and the Robertson–Seymour graph-minor theorem~\cite{RobertsonSeymour2004} --- guarantee that many naturally occurring Chisari quivers are indeed WQO.

%-----------------------------------------------------------------
\begin{thebibliography}{99}
\bibitem{Higman1952} G.~Higman, \emph{Ordering by divisibility in abstract algebras}, Proc. London Math. Soc. (3) \textbf{2} (1952), 326–336.
\bibitem{Kruskal1960} J.~B. Kruskal, \emph{Well-quasi-ordering, the tree theorem, and Vazsonyi’s conjecture}, Trans. Amer. Math. Soc. \textbf{95} (1960), 210–225.
\bibitem{Rosenstein1982} J.~G. Rosenstein, \emph{Linear Orderings}, Academic Press, 1982.
\bibitem{RobertsonSeymour2004} N.~Robertson and P.~D. Seymour, \emph{Graph minors. XX. Wagner’s conjecture}, J. Combin. Theory Ser. B \textbf{92} (2004), 325–357.
\bibitem{Harry1995} H.~Alt, \emph{The cover digraph of a quasi-order}, Order \textbf{12} (1995), 201–217.
\bibitem{Altomare2020} C.~Altomare, \emph{Transitive reductions of quasi-orders and their applications}, Order \textbf{37} (2020), 421–443.
\end{thebibliography}

\end{document}
