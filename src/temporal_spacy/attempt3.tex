\documentclass{article}
\usepackage{amsmath,amssymb,amsthm}
\usepackage{enumitem}

%-----------------------------------------------------------------
% theorem‑like environments
%-----------------------------------------------------------------
\newtheorem{definition}{Definition}
\newtheorem{axiom}{Axiom}
\newtheorem{proposition}{Proposition}
\newtheorem{theorem}{Theorem}
\newtheorem{example}{Example}
\newtheorem{remark}{Remark}

\title{Thread--Equivalence Posets (TE--posets) \& Rail--Switch Posets}
\author{}
\date{}

\begin{document}
\maketitle

%=================================================================
\section{Core notion}

\begin{definition}[TE‑poset]\label{def:TE}
A \emph{thread--equivalence poset} (TE‑poset) is a triple
\[
  \mathcal P \;=\;(P,\le,T)
\]
consisting of a set of vertices $P$, a preorder $\le$ on $P$, and a
partition $T$ of~$P$ into \emph{threads}, subject to the axioms
\begin{enumerate}[label=\textbf{T\arabic*}, wide = 0pt]
  \item\label{ax:t1} (\textbf{Thread partition})
        $T$ is a partition of $P$.  For $x\in P$ let
        $\operatorname{thr}(x)$ denote the block containing~$x$.
  \item\label{ax:t2} (\textbf{Linearity inside threads})
        For every thread $S\in T$ the induced order $(S,\le)$ is a chain,
        i.e. $(\forall x,y\in S)[x\le y\ \text{or}\ y\le x]$.
  \item\label{ax:t3} (\textbf{No strict cross‑thread order})
        If $x\le y$ and $\operatorname{thr}(x)\neq\operatorname{thr}(y)$ then
        $y\le x$ also holds.  In other words, \emph{every cross‑thread
        comparison is an equivalence}.
\end{enumerate}
Write $x\sim y$ if $x\le y\le x$.
\end{definition}

\begin{remark}
Collapsing each thread to a point turns a TE‑poset into a
\emph{weak order} (a chain of antichains).  The additional information
retained by a TE‑poset is the internal chain structure of each thread
and the selective cross‑thread equivalences.
\end{remark}

%=================================================================
\section{Rank‑aligned TE‑posets (rail--switch posets)}

The `rail--switch' diagrams the authors have in mind demand that every
cross‑thread equivalence join vertices that live on the \\emph{same
rank}.  We therefore strengthen the basic TE‑axioms by a single
alignment condition.

\begin{definition}[RS‑poset]\label{def:RS}
A \emph{rail--switch poset} (RS‑poset) is a TE‑poset
$\mathcal P=(P,\le,T)$ equipped with a \emph{rank function}
$\rho:P\to\mathbb N$ such that
\begin{enumerate}[label=\textbf{T4}, wide=0pt]
 \item\label{ax:t4} (\textbf{Rank alignment})  For all $x<y$ in the same
   thread $\rho(x)<\rho(y)$, and for every cross‑thread equivalence
   $x\sim y$ we have $\rho(x)=\rho(y)$.
\end{enumerate}
The data $(P,\le,T,\rho)$ will be written $(P,\le,T,\rho)$.
\end{definition}

\begin{remark}
Axioms \ref{ax:t1}--\ref{ax:t4} force the covering graph of an
RS‑poset to consist \emph{exclusively} of vertical in‑thread edges and
horizontal cross‑thread edges; cf. Fig.~\ref{fig:rail-switch}.  The
``rail--switch'' ASCII motifs are therefore not special cases but the
\emph{only} possible local pictures.
\end{remark}

%-----------------------------------------------------------------
\subsection{Canonical rail--switch embedding}

\begin{proposition}[Grid embedding]\label{prop:RSplanar}
Enumerate the threads $T=\{S_0,\dots,S_{m-1}\}$ and position each
vertex $v\in S_i$ at $(x,y)=(i,\rho(v))\in\mathbb Z^2$.  Draw
\emph{thread edges} vertically and \emph{equivalence edges}
horizontally.  The resulting drawing is planar; it is unique up to
isotopy and coincides with the intended rail--switch diagram.
\end{proposition}

\begin{proof}
By \ref{ax:t4} all horizontal edges lie on distinct $y$‑coordinates and
all vertical edges on distinct $x$‑coordinates.  Hence no two edges
intersect in their interiors.
\end{proof}

\begin{theorem}[Characterisation]
For finite structures the following are equivalent:
\begin{enumerate}[label=\textup{(\alph*)}]
  \item a poset is an RS‑poset;
  \item its Hasse diagram admits a planar straight‑line drawing whose
        edges are \emph{all} horizontal or vertical and such that every
        vertical line supports a chain.
\end{enumerate}
\end{theorem}

\begin{proof}
$(a)\Rightarrow(b)$ is Proposition~\ref{prop:RSplanar}.  For
$(b)\Rightarrow(a)$, take the vertical lines as threads, order them by
the $y$‑coordinate, and read horizontals as equivalences.  The
resulting structure satisfies T1--T4.
\end{proof}

%=================================================================
\section{Admissible mutations for RS‑posets}

The primitive operations from Definition~\ref{def:ops} remain sound
provided they respect the rank map $\rho$:
\begin{itemize}
  \item \textsc{addThread}: choose ranks disjoint from existing
        $\rho$‑values; e.g. append the new thread to the right.
  \item \textsc{insertBetween}: when splicing $z$, set
        $\rho(z)=\rho(x)+1$ and increment the ranks of all higher
        vertices in that thread.
  \item \textsc{formEquiv}: only allowed if $\rho(x)=\rho(y)$.
  \item \textsc{breakEquiv}: always safe (it removes edges only).
\end{itemize}
Each operation preserves T1--T4 and therefore closes the class of
RS‑posets.

%=================================================================
\section{Connections to existing structures}\label{sec:related}

\begin{proposition}[Translations]
There are bijective correspondences between finite RS‑posets and each
of the following well‑known models:
\begin{enumerate}[label=\textup{(\roman*)}]
  \item \emph{Pomsets with a chain partition} (Pratt, Gischer), which
        underpin Message Sequence Charts;
  \item the dependency posets of \emph{Mazurkiewicz traces} in
        concurrency theory;
  \item \emph{heaps of pieces} (Viennot, Stembridge) whose concurrency
        relation is ``same rank''.
\end{enumerate}
\end{proposition}

\begin{proof}[Sketch]
Collapse each thread to a label, keep the rank as height, and record
horizontal two‑cycles as synchronisation or heap contact edges.  The
converse reconstruction splits each labelled event back into per‑thread
copies.  Axioms T1--T4 translate exactly to the defining rules in the
cited frameworks.
\end{proof}

For the reader's convenience Table~\ref{tab:dictionary} summarises the
terminology.

\begin{table}[h]
  \centering
  \begin{tabular}{l|l|l}
    RS‑poset & Pomset/MSC & Trace / Heap \\
    \hline
    thread & lifeline & process / piece kind \\
    rank $\rho$ & time slice & lattice level \\
    $x\sim y$ & synchronous message & concurrency edge
  \end{tabular}
  \caption{Dictionary of notions}
  \label{tab:dictionary}
\end{table}

%=================================================================
\section{Example revisited}

The earlier example with $a\sim d$ and $c\sim e$ \emph{violates}
T4 because $\rho(c)\neq\rho(e)$.  It is therefore a TE‑poset that is
\emph{not} an RS‑poset.  A valid RS instance is instead
\[
  a\sim d< b < c\sim e,
\qquad
  \rho(a)=\rho(d)=0,\;\rho(b)=1,\;\rho(c)=\rho(e)=2.
\]
Its drawing is the canonical rail--switch
\[
  \begin{array}{c}
    a\longleftrightarrow d \\
    |\quad\;\;\quad| \\
    b\quad\;\;\quad  \\
    |\quad\;\;\quad| \\
    c\longleftrightarrow e
  \end{array}
\]

%=================================================================
\section{Outlook}

Future work includes algorithmic questions (rank maintenance under
mutation, counting RS‑posets) and categorical semantics (viewing
RS‑posets as string diagrams for synchronising processes).

\end{document}
